\section{Thesis Statement}
NDN faces significant challenges in producer mobility scenarios. When a producer moves, existing mechanisms inevitably require a series of steps: the producer must re-register its data prefix, the Named-data Link State Routing (NLSR) protocol must propagate global routing updates, routers update forwarding entries, and consumers retransmit their Interests to re-establish connectivity. This sequence introduces considerable latency and packet loss, undermining user experience and reliability, particularly for latency-sensitive or real-time applications.

\textbf{I assert that these mobility challenges can be effectively addressed through directed, scoped, flood-based approaches.} 
Flooding refers to selectively broadcasting interest or \DataPackets\ to multiple neighbouring nodes simultaneously, thereby rapidly establishing temporary communication paths without relying on slow NLSR route convergence. By enabling immediate local discovery of relocated producers, flooding can significantly reduce latency and packet loss during handoffs, effectively bridging the communication gap until stable global routing updates are complete.

To demonstrate this claim, my solutions tackle the mobility problem through three incremental stages, each addressing specific limitations of NDN's existing mobility handling mechanisms:

\begin{itemize}
    \item \textbf{Stage I Data Flooding for Pending Interests:} Immediately after relocation, the producer floods the \DataPackets\ to proactively satisfy outstanding interests issued before its move, thereby avoiding data loss caused by outdated forwarding entries.
    \item \textbf{Stage II Interest Flooding for Temporary Channel Establishment:} To handle new interests arriving during the slow convergence period, controlled interest flooding is introduced. This mechanism rapidly discovers the producer's new location and establishes temporary bidirectional communication paths, ensuring continuous data flow despite ongoing routing updates.
    \item \textbf{Stage III Localised Flooding with Prefix Indexing:} Recognising the scalability limits of flooding, this final stage incorporates localised flood boundaries and lightweight prefix indexing techniques. By guiding interest and \DataPackets\ only towards the most relevant regions, this approach significantly reduces network overhead, enabling the flooding-based solution to scale efficiently to larger and more dynamic network environments.
\end{itemize}

Through comprehensive analysis and experiment evaluation, I will quantitatively measure improvements in key performance metrics such as handoff latency, packet loss rates, and network overhead. The results will validate that strategically controlled flooding can offer effective and efficient producer mobility support in diverse NDN deployment scenarios.
